\documentclass[a4paper]{book}
\usepackage{geometry}
\usepackage{amsmath}
\usepackage{fontspec}
\usepackage{polyglossia}

\geometry{a4paper,left=2cm,right=2cm,top=1cm,bottom=1cm}
\newtheorem{theorem}{定理}[chapter]
\newtheorem{lemma}{命题}[chapter]
\newtheorem{proof}{证明}
\newtheorem{corollary}{推论}[chapter]

\title{随机变量}
\author{toorevitimirp}
\setmainfont[]{Noto Serif CJK SC}
\begin{document}
\chapter{}
\chapter{}
\chapter{}
\chapter{随机变量}
    \section{随机变量}
        随机变量、分布列、分布函数的定义
    \section{离散型随机变量}
    \section{期望}
        \begin{equation*}
            E[X] = \sum_{x:p(x)>0}xp(x)
        \end{equation*}
    \section{随机变量函数的期望}
        \begin{lemma}
            如果X是一个离散型随机变量,其取值可能为$x_i$,i$\geq$1,相应的取值概率为$p(x_i)$,
            
            那么,对任一实值函数,都有
            \begin{equation*}
                E[g(X)] = \sum_ig(x_i)p(x_i)
            \end{equation*}
        \end{lemma}

        \begin{corollary}
            \begin{equation*}
                E[aX + b] = aE[X] + b
            \end{equation*}
        \end{corollary}

    \section{方差}
        \begin{equation*}
            Var(X) = E[X^2] - (E[X])^2
        \end{equation*}
        标准差是方差的平方根
    \section{伯努利随机变量和二项随机变量}
        参数为(n,p)的二项随机变量的分布列:
        \begin{equation*}
            p(i) = \binom{n}{i}p^i(1-p)^{n-i}
        \end{equation*}
        由二项式定理,
        \begin{equation*}
            \sum^\infty_{i=0}p(i) = 1
        \end{equation*}
        二项随机变量的期望:
        \begin{equation*}
            E[X] = np
        \end{equation*}
        方差:
        \begin{equation*}
            Var(X) = np(1-p)
        \end{equation*}
        二项函数的分布函数的递推公式:
        \begin{equation*}
            P\{X = k + 1\} = \frac{p}{1-p} \frac{n-k}{k+1}P\{X = k\}
        \end{equation*}
        \begin{theorem}
            如果X是一个参数为$(n,p)$的n二项随机变量,$0<p<1$。那么当$k$从$0$到$n$时,$P{X=k}$
            
            一开始单调递增,然后一直单调递减,当$k=\lfloor(n+1)p\rfloor$时取最大值。
        \end{theorem}
    \section{泊松随机变量}
        定义:对于$\lambda > 0,i = 0,1,2,3\dots$
        \begin{equation*}
            p(i) = P\{X = i\} = e^{-\lambda}\frac{\lambda^i}{i!}
        \end{equation*}
        求和:
        \begin{equation*}
            \sum_{i=0}^\infty p(i) = e^{-\lambda} \sum_{i=0}^\infty\frac{\lambda^i}{i!} = e^{-\lambda}e^\lambda = 1
        \end{equation*}
        期望与方差:
        \begin{equation*}
            Var(x) = E[X] = \lambda
        \end{equation*}
        递推公式:
        \begin{equation*}
            \frac{P\{X = i + 1\}}{P\{X = i\}} = \frac{e^{-\lambda}\lambda^{i+1}/(i+1)!}{e^{-\lambda}/i!} = \frac{\lambda}{i+1}
        \end{equation*}
        泊松分布成立的假设条件
        \linebreak
        \linebreak
        当二项分布的n很大p很小时,二项分布与$\lambda = np$的泊松分布近似。
    \section{其他离散型概率分布}
        \subsection{几何随机变量}
            独立重复试验中,每次成功的概率为p(0<p<1),重复试验直到首次成功。令X表示需要试验的次数.
            \begin{equation*}
                P\{X = n\} = (1 - p)^{n-1}p
            \end{equation*}
        期望:
            \begin{equation*}
                E[X] = \frac{1}{p}
            \end{equation*}
        方差:
        \begin{equation*}
            Var(X) = \frac{1-p}{p^2}
        \end{equation*}

        \subsection{负二项随机变量}
        独立重复试验中,每次成功的概率为p,0<p<1,试验持续进行直到试验累计成功r次。令X表示试验的总次数。
        \begin{equation*}
            P\{ X = n \} = \binom{n-1}{r-1}p^r(1-p)^(n-r) \quad
            n = r,r+1 \dots
        \end{equation*}
        期望:
        \begin{equation*}
            E[x] = \frac{r}{p}
        \end{equation*}
        方差:
        \begin{equation*}
            Var(X) = \frac{r(1-p)}{p^2}
        \end{equation*}
        X的参数为(r,p)
        \linebreak
        几何随机变量是参数为(1,p)的负二项随机变量
    \subsection{超几何随机变量}
    一个坛子里有N个球,m个白球,N-m个黑球,从中随机(无放回)取出n个球,令X表示取出来的白球数。
    \begin{equation*}
        P\{X = i \} = \frac{\binom{m}{i}\binom{N-m}{n-i}}{\binom{N}{n}}
    \end{equation*}
    期望:
    \begin{equation*}
        E[x] = \frac{mn}{N}
    \end{equation*}
    方差:
    \begin{equation*}
        Var(X) = \frac{mn}{N}[\frac{(n-1)(m-1)}{N-1} + 1 - \frac{mn}{N}]
    \end{equation*}
    当m和N很大时,负二项随机分布近似为参数为$(n,\frac{m}{N})$的二项分布。
    \subsection{$\zeta$ 分布}
    \section{极大似然估计}
    \section{随机变量和的期望}
    一组随机变量和的期望等于他们各自期望的和
    \begin{equation*}
        E[\sum_{i=1}^mX_i] = \sum_{i=1}^nE[X_i]
    \end{equation*}
    \section{分布函数的性质}
    \begin{align}
        F(a) \leq F(b)&,a<b \\
        \lim_{b \to \infty} F(b) &= 1 \\
        \lim_{b \to -\infty} F(b) &= 0 \\
        P\{a<X\leq b\} &= F(b) - F(a) \\
        \begin{split}
            P\{X<b\} = P(\lim_{n \to \infty}\{X \leq b - \frac{1}{n}\}) 
            &= \lim_{n \to \infty}P(X \leq b - \frac{1}{n})
            =\lim_{n \to \infty }F(b - \frac{1}{n}) 
        \end{split}
    \end{align}
    
\end{document}